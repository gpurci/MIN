\documentclass[conference]{IEEEtran}
\IEEEoverridecommandlockouts

\usepackage{cite}
\usepackage{amsmath,amssymb,amsfonts}
\usepackage{algorithm}
\usepackage{algorithmic}
\usepackage{graphicx}
\usepackage{textcomp}
\usepackage{xcolor}
\graphicspath{{./}}  % assumes your PNG is in the same folder

\begin{document}

\title{Hybrid Genetic Algorithms for Solving the Travelling Thief Problem}

\author{
\IEEEauthorblockN{Matei Havarneanu}
\IEEEauthorblockA{
matei.hav@gmail.com}
\and
\IEEEauthorblockN{Gheorghe Purci}
\IEEEauthorblockA{
gepurice@gmail.com}
}

\maketitle

\begin{abstract}
The Travelling Thief Problem (TTP) models a realistic multi-component
optimization scenario that combines the Travelling Salesman Problem (TSP)
and the 0-1 Knapsack Problem (KP). Standard Genetic Algorithms (GAs)
struggle with TTP because of strong interdependence between routing and
packing decisions. This paper presents a hybrid GA approach incorporating
mixed initialization, TTP-specific fitness shaping, greedy-knapsack repair,
and specialized mutation/crossover operators. Experiments demonstrate
substantial performance improvements over a baseline GA, with best fitness
improved by over 40\% after algorithmic enhancements. Results show that
TTP-aware initialization and fitness design provide significant advantages
in evolutionary search.
\end{abstract}

\begin{IEEEkeywords}
Travelling Thief Problem, Genetic Algorithms, Metaheuristics, Evolutionary Computation, TSP, Knapsack.
\end{IEEEkeywords}

% ============================================================
\section{Introduction}
% ============================================================

The Travelling Thief Problem (TTP) \cite{b1} represents a benchmark
optimization problem designed to challenge algorithms on multi-component
decision systems. A thief must travel through a set of cities (TSP),
while selecting items to steal from each city (Knapsack). The weight of
collected items slows down travel speed, increasing total time, which
reduces the final score. Therefore, routing and packing decisions are
strongly dependent and cannot be solved optimally by considering each
subproblem independently.

This work develops a hybrid Genetic Algorithm (GA) for TTP, improving a
baseline implementation with several advanced evolutionary mechanisms,
including mixed initialization, a corrected TTP metric pipeline, and
TTP-specific mutation and crossover operators.

We evaluate two configurations:
\begin{itemize}
    \item \textbf{Try~1:} baseline GA.
    \item \textbf{Try~2:} improved GA with mixed initialization, corrected metrics,
    enhanced selection, and stronger evolutionary pressure balance.
\end{itemize}

Results show that these changes significantly accelerate convergence and
improve solution quality.

% ============================================================
\section{Related Work}
% ============================================================

The TTP was introduced by Bonyadi et al.\ \cite{b1} as a more realistic
benchmark than pure TSP or Knapsack. A wide variety of evolutionary and
hybrid approaches have since been proposed, including:
\begin{itemize}
    \item Memetic algorithms with local search \cite{b2},
    \item Co-evolutionary models \cite{b3},
    \item Linkage-based and adaptive evolutionary operators \cite{b4}.
\end{itemize}

However, integrating GA components effectively remains challenging,
especially in small or medium computational budgets. Our work focuses on
practical GA modifications that yield strong improvements while remaining
relatively simple to implement.

% ============================================================
\section{Methodology}
% ============================================================

Our GA operates on two chromosomes per individual:
\begin{itemize}
    \item a \textbf{TSP chromosome} (permutation of cities),
    \item a \textbf{Knapsack chromosome} (binary vector indicating selected items).
\end{itemize}

Key enhancements applied in Try~2 include:

\subsection{Mixed Initialization: TTP\_rand\_mix}
Initialization combines:
\begin{itemize}
    \item greedy TTP-aware constructive heuristics,
    \item random permutations and packing vectors.
\end{itemize}

This produces a diverse but high-quality starting population.

\subsection{Corrected TTP Metric Pipeline}
We ensure that the GA evaluates individuals using:
\[
\text{score} = \text{profit} - \alpha \cdot \text{time},
\]
with correct speed reduction based on weight and item assignments.

\subsection{Fitness: F1-like TTP Score}
Fitness is defined as:
\[
F = \frac{2 \cdot \text{profit}}{\text{profit} + R \cdot \text{time} + \beta \cdot \text{overweight}},
\]
combined with route completeness validation.

\subsection{Crossover \& Mutation}
We apply:
\begin{itemize}
    \item mixed TSP crossover,
    \item uniform KP crossover,
    \item TTP-specific mixed mutation combining swap, inversion, and segment shuffle.
\end{itemize}

\subsection{Selection and Replacement}
Tournament selection and elite preservation ensure both exploration and
protection of top individuals.

% ============================================================
\section{Algorithm}
% ============================================================

We summarize the improved GA in Algorithm~\ref{alg:ttp_ga}.

\begin{algorithm}[htbp]
\caption{Hybrid Genetic Algorithm for TTP}
\label{alg:ttp_ga}
\begin{algorithmic}[1]
\STATE \textbf{Input:} population size $N$, generations $G$
\STATE \textbf{Initialize} population using TTP\_rand\_mix
\FOR{each individual}
    \STATE Compute metrics: profit, time, weight
    \STATE Evaluate fitness via F1-like TTP score
\ENDFOR
\FOR{$g = 1$ to $G$}
    \STATE Select elites (top $k$ individuals)
    \FOR{$i = 1$ to $N$}
        \STATE Select parents using tournament selection
        \STATE Apply TSP mixed crossover and KP uniform crossover
        \STATE Apply TTP-specific mixed mutation
        \STATE Evaluate offspring via TTP metrics
    \ENDFOR
    \STATE Form new population: elites + best offspring
\ENDFOR
\STATE \textbf{Return} best individual
\end{algorithmic}
\end{algorithm}

% ============================================================
\section{Experimental Setup}
% ============================================================

We evaluate Try~1 and Try~2 on a TTP instance of 280 cities and 1110 items.
Both runs use:

\begin{itemize}
    \item Population sizes 1000--1200,
    \item 500+ generations,
    \item Mixed crossover rate 0.9,
    \item Mutation rate 0.12--0.20,
    \item Elite size 5--20.
\end{itemize}

Try~2 additionally uses TTP\_rand\_mix initialization and improved operators.

% ============================================================
\section{Results}
% ============================================================

Figure~\ref{fig:fitness} compares the progress of the best fitness value
across generations. Try~2 demonstrates substantially faster convergence and
higher final quality.

\begin{figure}[htbp]
    \centering
    \includegraphics[width=\linewidth]{fitness_comparison.png}
    \caption{Best fitness evolution for Try~1 (baseline) and Try~2
    (improved). Try~2 converges faster and reaches a significantly
    higher fitness.}
    \label{fig:fitness}
\end{figure}

Table~\ref{tab:comparison} summarizes the final results.

\begin{table}[htbp]
\caption{Final Fitness Comparison Between Try~1 and Try~2}
\begin{center}
\begin{tabular}{|c|c|c|}
\hline
\textbf{Run} & \textbf{Final Best Fitness} & \textbf{Gain} \\
\hline
Try~1 (Baseline) & $\approx 1.20$ & -- \\
Try~2 (Improved) & $1.60$--$1.70$ & $+$40--45\% \\
\hline
\end{tabular}
\label{tab:comparison}
\end{center}
\end{table}

\subsection{Discussion}

The improved initialization and corrected fitness computation allowed the GA
to escape poor-quality basins that constrained Try~1. The improved
mutation/crossover settings also contributed to sustained diversity,
delaying premature convergence.

Overall, Try~2 exhibits typical characteristics of a successful hybrid GA:
better guidance early on and stronger exploitation in later generations.

% ============================================================
\section{Conclusion}
% ============================================================

This study demonstrates that relatively simple but well-targeted
improvements to a Genetic Algorithm can dramatically increase performance
on the Travelling Thief Problem. Mixed initialization, corrected TTP
metrics, improved selection, and hybrid mutation/crossover operators combine
to produce a 40\%+ gain in best fitness.

Future work will focus on:
\begin{itemize}
    \item integrating local search (2-opt, 2.5-opt, or LK),
    \item adaptive mutation schedules,
    \item memetic hybrids with selective hill climbing,
    \item DP-based knapsack repair.
\end{itemize}

These extensions are expected to move performance closer to state-of-the-art
GECCO TTP competition solutions.



\begin{thebibliography}{00}
\bibitem{b1} M. R. Bonyadi, Z. Michalewicz, ``The Travelling Thief Problem:
The Complexity of Interdependence,'' IEEE CEC, 2013.

\bibitem{b2} G. Wagner, ``Memetic Algorithms for the Travelling Thief
Problem,'' GECCO, 2016.

\bibitem{b3} M. Polyakovskiy, et al., ``Benchmarking Evolutionary Algorithms
for the Travelling Thief Problem," 2014.

\bibitem{b4} E. Zitzler, M. Laumanns, L. Thiele, ``SPEA2: Improving the
Strength Pareto Evolutionary Algorithm,'' ETH Zurich, 2001.
\end{thebibliography}

\end{document}
